\chapter{Auswertung und Laufzeitmessung}
\label{chapter:auswertung}
In diesem Kapitel wird die Performance des Programms zur Lösung des Pressenplanungsproblems ausgewertet.
Dazu werden zunächst folgende Konfigurationen des Programms miteinander verglichen:
\begin{itemize}
    \item SMT mit Incremental Refinement, \texttt{QF\_LRA}, \texttt{OpenSMT}
    \item SMT mit Incremental Refinement, \texttt{QF\_LIA}, \texttt{OpenSMT}
    \item SMT mit Incremental Refinement, \texttt{QF\_LIA}, \texttt{Z3}
    \item OMT mit linearer Optimierung, \texttt{QF\_LIA}, \texttt{Z3}
\end{itemize}

Die Wahl der Logiken \texttt{QF\_LIA} und \texttt{QF\_LRA} wurde bereits in Abschnitt \ref{sec:sortpoly} diskutiert.
Für beide Logiken wird der jeweilige Gewinner jener Kategorien der SMT-COMP 2024 \texttt{OpenSMT} (Version 2.7.0) verwendet \cite{smtcomp2024results}.
Zusätzlich wird auch \texttt{Z3} (Version 4.13.0) mit Incremental Refinement und Logik \texttt{QF\_LIA} getestet.
Trotz schwachem Ergebnis in der SMT-COMP 2024 erwies sich diese Kombination in einigen Testfällen als hoch-performant.
Aufgrund dieser Erkenntnis wurde auch die Performance weiterer Solver untersucht.
Diese Untersuchung ergab allerdings keine nennenswerten Ergebnisse, sodass diese im Folgenden ausgelassen werden.
Weiter wird auch eine Programmkonfiguration mit \texttt{Z3} und OMT in Logik \texttt{QF\_LIA} evaluiert.

Danach wird die Performance der insgesamt besten Programmkonfiguration mit jener von Germanedge's Heuristik verglichen und ausgewertet.
Am Ende werden mögliche Verbesserungen diskutiert.

\section{Vorbetrachtung}
Die Laufzeitmessungen der Konfigurationen des in dieser Arbeit entwickelten Programms erfolgen auf einer Maschine mit
acht CPU-Kernen mit Takt von jeweils 2,4 GHz und 16 GB Arbeitsspeicher.
Aufgrund einiger Firmenregularien kann die Laufzeitmessung der Heuristik nicht auf der gleichen Maschine wie jener des hier entwickelten Programms erfolgen.
Die firmeninterne Maschine weist allerdings sehr ähnliche Hardware-Spezifikationen auf, sodass ein Vergleich dennoch aussagekräftig ist.
Die Laufzeitmessungen erfolgen unter ähnlichen Bedingungen mit minimaler Auslastung durch andere Prozesse.

Insgesamt wird die Lösung von 26 verschiedenen Probleminstanzen verglichen.
Darunter befinden sich sowohl reale als auch konstruierte Testfälle.
Zu Zwecken der Anonymisierung werden diese allerdings nicht als solche gekennzeichnet.

Ausgewertet wird die Güte der Lösungen nach den Optimierungszielen aus Abschnitt \ref{sec:optimierung}.
Zusätzlich kommt als viertes lexikografisches Optimierungsziel die Laufzeit des Programms beziehungsweise der Heuristik hinzu.
Sind alle vier Metriken zweier Lösungen einer Probleminstanz gleich, dann werden die Lösungen als gleich gut bewertet.

Die Laufzeitmessungen werden in den folgenden Abschnitten mit verschiedenen oberen Laufzeitgrenzen (Timeouts) versehen.
Gemessen wird die Zeit, welche Programm oder Heuristik für eine endgültige Lösung benötigen.
Ermittelt das Programm also beispielsweise eine Lösung nach $30s$, rechnet aber bis zum Timeout weiter und verbessert diese Lösung nicht,
dann wird die Messung der Laufzeit mit \textit{\gls{TO}} (kurz für Timeout) versehen.
Die gemessene Laufzeit des Programms schließt zwar alle Schritte von Eingabeverarbeitung über Kodierung der Constraints bis hin zur Lösung durch den Solver ein,
umfasst letztendlich aber fast ausschließlich die Laufzeit des Solvers.

Folgend die Erklärung aller Kriterien der Vergleiche zur Lösung einer Probleminstanz:
\begin{table}[H]
    \centering
    \begin{tabular}{|l|l|}
        \hline
        \textbf{Kriterium} & \textbf{Erklärung} \\
        \hline
        Nr                         & Identifizierende Nummer der Probleminstanz \\
        $\lvert C \rvert$          & Anzahl der Komponenten \\
        $\lvert C_{Spec} \rvert$   & Anzahl verschiedener Komponenten-Spezifikationen \\
        Typ                        & SMT (inkrementelles Verfeinern) oder OMT (lineare Optimierung) \\
        Logik                      & \texttt{QF\_LIA} oder \texttt{QF\_LRA} \\
        Solver                     & Externer SMT-Solver \\
        $p$                        & Minimale Pressen-Anzahl (erklärt in Abschnitt \ref{sec:parallelisierung}) \\
        Größe                      & Anzahl symbolischer Operationen für kleinste erfüllbare $\lvert P \rvert$, sonst $p$ \\
        \#$\mathbb{B}$             & Anzahl boolescher Unbekannter im Constraint-System \\
        \#$\mathbb{Z}$             & Anzahl numerischer Unbekannter im Constraint-System \\
        Zeit in s                  & Laufzeit in Sekunden (aufgerundet), TO ist Timeout \\
        $\lvert C_{Not} \rvert$    & Anzahl unverplanter Komponenten \\
        $\lvert P \rvert$          & Anzahl geplanter Pressen \\
        Rest in mm                 & Gesamtrest in Millimetern \\
        Zert.                      & \cmark \; falls Lösung zertifiziert optimal ist, \xmark \; sonst \\
        \hline
    \end{tabular}
    \caption{Kriterien für die Laufzeitmessungen in Anhang \ref{appendix:results}}
    \label{table:vglkriterien}
\end{table}

\section{Vergleich verschiedener Programm-Konfigurationen}
In diesem Abschnitt wird die Performance verschiedener Konfigurationen des in dieser Arbeit entwickelten Programms verglichen.
Der Timeout für alle Probleminstanzen beträgt zehn Minuten, die maximal tolerierte Laufzeit aus Sicht des Kunden.
Die tabellarische Auswertung der Laufzeitmessungen ist in Anhang \ref{appendix:results}, Tabelle \ref{tab:vglkodierungreal} zu finden.

Die Ergebnisse zeigen, dass alle Programmkonfigurationen korrekte und auch optimale Lösungen für bestimmte Eingaben finden.
Von insgesamt 26 Testfällen werden 25 gelöst.
Zehn Testfälle können innerhalb von zehn Minuten sogar optimal gelöst und zertifiziert werden.
Eine optimale Lösung ist genau dann zertifiziert, wenn der Solver nachweisen kann, dass keine bessere Lösung existiert.

Die Kodierung des Constraint-Systems in den Logiken \texttt{QF\_LIA} und \texttt{QF\_LRA} liefert Lösungen ähnlicher Güte.
Wenn die Lösung mit Kodierung \texttt{QF\_LIA} besser ist, dann ist sie das deutlich in Bezug auf den Rest (Testfall 6).

Erstaunlich ist die dominante Performance von \texttt{Z3} mit Kodierung in \texttt{QF\_LIA} und Optimierung durch inkrementelles Verfeinern.
In fast allen Testfällen liefert diese Konfiguration die beste Lösung.
Besonders bemerkenswert sind dabei die Testfälle 2, 6, 18 und 20.
Die Konfiguration liefert dort als einzige die zertifiziert optimalen Lösungen.
Diese Ergebnisse stehen in Konflikt zu jenen der SMT-COMP 2024, in welcher \texttt{Z3} klar hinter \texttt{OpenSMT} einzuordnen ist \cite{smtcomp2024results}.
Höchstwahrscheinlich liegt das am niedrigen Verhältnis arithmetischer Unbekannter zu booleschen Unbekannten in der Modellierung des Pressenplanungsproblems.
In den meisten Benchmarks der SMT-COMP 2024 für \texttt{QF\_LIA} sind die Unbekannten fast ausschließlich von der Sorte \texttt{Int}.

Die Konfiguration mit der Optimierung nach linearen Verfahren in OMT, \texttt{Z3} und \texttt{QF\_LIA}
liefert für einfache Testfälle wie 1,2,3 oder 5 ebenfalls optimale Ergebnisse, für größere Testfälle allerdings gar keine.
Das ist durch interne Beschränkungen von \texttt{Z3}'s Optimierungsmethodik und das Fehlen der Ausgabe von Zwischenergebnissen zu erklären.

Betrachten wir nun die beste Konfiguration mit inkrementellem Verfeinern, Kodierung in \texttt{QF\_LIA} und Solver \texttt{Z3} etwas genauer.
Das Finden und Zertifizieren der optimalen Lösung scheint für Probleminstanzen, bei welchen eine Lösung ohne Rest existiert am einfachsten.
Testfall 20 zeigt, dass das sogar für größere Eingaben innerhalb von zehn Minuten möglich ist.
Hingegen deuten die Testfälle 4, 13 und 17 darauf hin, dass der Nachweis von Optimalität selbst für kleine Testfälle,
bei denen keine Lösung mit Rest null existiert, sehr schwer ist.
Das ist so, weil der Solver erst nachweisen muss, dass keine andere Lösung existiert, die besser ist.
Für den Fall mit Rest null entsteht ein einfacher Widerspruch zu Constraint \ref{listing:constraintpositivewaste} in Abschnitt \ref{sec:kodierungconstraints}.
Bei allen anderen Fällen mit optimalem Rest $w$ und $0 < k \leq w$ muss für alle möglichen $w$ besseren Lösungen mit Rest $k$ Unerfüllbarkeit nachgewiesen werden.

Die Lösung der größten Probleminstanzen 11, 12, 25 und 26 ist mit Betrachtung des Rests vermutlich weit vom Optimum entfernt.
Unter allen Testfällen bleibt ein Testfall (12) innerhalb von zehn Minuten sogar gänzlich ungelöst.
Testfälle wie dieser mit über 250 Bauteilen und mehr als acht Pressen stellen die größtmöglichen und damit schwierigsten Eingaben aus der Realität dar.
Auch, wenn sowohl Anzahl Unbekannter und Formeln, als auch Größe letzterer linear zur Eingabe steigen (Abschnitt \ref{sec:modellierung}),
scheint die Schwere des Problems mit erhöhter Komplexität des Constraint-Systems weitaus stärker zu wachsen.
Große Probleme vergrößern nicht nur den Lösungsraum, sondern erhöhen auch die Anforderungen des Solvers an die Hardware.
Laufzeitmessungen auf einer Maschine mit 128 GB Arbeitsspeicher und Timeout nach zehn Stunden (Anhang \ref{appendix:results}, Tabelle \ref{tab:vglkodierungliberal}) zeigen für Probleme
11, 12, 25 und 26 Arbeitsspeichernutzung von bis zu 40 GB pro Problem.
Je länger der Solver-Prozess läuft, desto mehr Speicher benötigt er, weil er Lemmas lernt (CDCL(T), Abschnitt \ref{sec:smtbasics}).

Während auch größere Probleme wie Testfall 24 ohne Rest innerhalb von zehn Stunden optimal gelöst werden können, kann selbst für den kleinen Testfall 4 nach zehn Stunden
noch immer keine optimale Lösung gefunden werden.
Testfälle 11 und 12 demonstrieren, dass die Solver mit mehr Laufzeit auch für die größten Probleme signifikant bessere Lösungen finden.
Auch dort bestätigt sich die Programmkonfiguration mit \texttt{Z3}, \texttt{QF\_LIA} und inkrementellem Verfeinern als performanteste.

\section{Vergleich mit Germanedge's Heuristik}
Der Vergleich der Laufzeitmessungen für Germanedge's Heuristik und das hier entwickelte Programm ist in Anhang \ref{appendix:results} Tabelle \ref{tab:vglheuristik} zu finden.
Das Programm findet innerhalb von zehn Minuten in 16 von 26 Testfällen die bessere Lösung.
In sieben von 26 Probleminstanzen siegt die Heuristik.
Bei den verbleibenden drei Testfällen finden Programm und Heuristik gleich gute Lösungen.

Die Lösungen der Heuristik enthalten fast immer deutlich weniger Rest als die des Programms.
Dennoch gewinnt das Programm knapp zwei Drittel der Testfälle, da die Minimierung des Rests lediglich Optimierungsziel 3 ist.
Aufgrund der lexikografischen Bewertung der Güte der Lösungen und der Wahl der Heuristik, gegebenenfalls wenige Bauteile ungeplant zu lassen oder mehr Pressen zu nutzen,
um einen geringeren Rest zu erzielen, kann das Programm in diesem Vergleich sehr gut überzeugen.
Während das Programm beispielsweise trotz 110 Metern Rest nach der lexikografischen Bewertung in Testfall 11 besser abschneidet, würde der Kunde höchstwahrscheinlich die
Lösung der Heuristik mit einem übrigen Bauteil und lediglich zwei Metern Rest bevorzugen.
Das deutet darauf hin, dass der Kunde anstelle der lexikografischen Bewertung der Lösung die Wahl einer Kostenfunktion in Betracht ziehen sollte.

An der Implementierung des Programms ändert das nichts.
Oberste Priorität stellt dann noch immer das Verplanen aller Komponenten dar.
Da das Programm bereits Kodierungen für $p$, $p+1$, $p+2$ und $p+3$ erstellt, enthält höchstwahrscheinlich eine dieser die optimale Lösung nach der gewünschten Kostenfunktion.

Die Heuristik kann besonders bei großen Eingaben ($\lvert C \rvert > 130$) im Vergleich zum Programm überzeugen.
Sie findet dort aufgrund einiger interner Vereinfachungen in geringer Zeit Lösungen nahe am Optimum.
Optimale Lösungen können von der Heuristik zwar möglicherweise gefunden, aber nicht als optimal zertifiziert werden.
Die Testfälle zeigen, dass das Programm das hingegen kann.
Für kleine und mittlere Testfälle ($\lvert C \rvert \leq 130$) sogar innerhalb von zehn Minuten, wenn Lösungen ohne Rest existieren.

Daher sollte das hier entwickelte Programm für den Einsatz in der Industrie bei kleineren bis mittleren Testfällen in Betracht gezogen werden.
Wenn die Restriktion an die Laufzeit liberaler gestaltet werden kann, also ein Timeout von zehn bis 24 Stunden möglich ist, dann kann das Programm selbst für
größere Probleme Anwendung in der Industrie finden.

\section{Diskussion}
Das in dieser Arbeit entwickelte Programm berücksichtigt nur die Kernelemente des Pressenplanungsproblems.
Die Heuristik implementiert bereits einige Sonderfälle wie Doppellagen in Pressen oder Pressen mit treppenartiger Abstufung.
Diese wären auch für das Programm wünschenswert.
Inwiefern die Eingliederung dieser in das Constraint-System sinnvoll ist, hängt davon ab,
wie die Modellierung dieser Sonderfälle die Komplexität des Constraint-Systems und damit die Laufzeit des Solvers beeinflusst.

Die Testfälle in Anhang \ref{appendix:results} Tabelle \ref{tab:vglkodierungreal} zeigen, dass das Constraint-System überschaubar viele Unbekannte kodiert.
Allerdings scheinen die Constraints groß und komplex.
Wenige zusätzliche, bestenfalls boolesche Unbekannte und dafür weniger komplexe Constraints könnten die Solver-Laufzeit positiv beeinflussen.
Außerdem verspricht eine mögliche Verbesserung der in Abschnitt \ref{sec:vorverarbeitung} betrachteten Funktion zur Ermittlung der Höchstanzahl an Schichten pro Presse großes Potenzial.
