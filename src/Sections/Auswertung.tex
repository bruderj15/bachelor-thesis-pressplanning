\chapter{Auswertung und Laufzeitmessung}
\label{chapter:auswertung}

%Was haben wir für Testfälle?
%Woher kommen die Testfälle?
%Wie wurden die Testfälle ausgewählt?
%Sind sie repräsentativ für reale Probleme?
%Vorgehen Laufzeitmessung erwähnen.
%Maschinen-Specs nennen.
%Vergleich von:

In diesem Kapitel wird die Performance des Programms zur Lösung des Pressenplanungsproblems ausgewertet.
Dazu werden zunächst folgende Konfigurationen des Programms miteinander verglichen:
\begin{itemize}
    \item SMT mit Incremental Refinement, \texttt{QF\_LRA}, \texttt{OpenSMT}
    \item SMT mit Incremental Refinement, \texttt{QF\_LIA}, \texttt{OpenSMT}
    \item SMT mit Incremental Refinement, \texttt{QF\_LIA}, \texttt{Z3}
    \item OMT mit linearer Optimierung, \texttt{QF\_LIA}, \texttt{Z3}
\end{itemize}

Die Wahl der Logiken \texttt{QF\_LIA} und \texttt{QF\_LRA} wurde bereits in Abschnitt \ref{sec:sortpoly} diskutiert.
Für beide Logiken wird der jeweilige Gewinner jener Kategorien der SMT-COMP 2024 \texttt{OpenSMT} (Version 2.7.0) verwendet \cite{smtcomp2024results}.
Zusätzlich wird auch \texttt{Z3} (Version 4.13.0) mit Incremental Refinement und Logik \texttt{QF\_LIA} getestet.
Trotz schwachem Ergebnis in der SMT-COMP 2024 erwies sich diese Kombination in einigen Testfällen als hoch-performant.
Aufgrund dieser Erkenntnis wurde auch die Performance weiterer Solver untersucht.
Diese Untersuchung ergab allerdings keine nennenswerten Ergebnisse, sodass diese im Folgenden ausgelassen werden.
Weiter wird auch eine Programmkonfiguration mit \texttt{Z3} und OMT in Logik \texttt{QF\_LIA} getestet.

Danach wird die Performance der insgesamt besten Programmkonfiguration mit jener von Germanedge's Heuristik verglichen und ausgewertet.
Am Ende werden mögliche Verbesserungen diskutiert.

\section{Vorbetrachtung}
Die Laufzeitmessungen der Konfigurationen des in dieser Arbeit entwickelten Programms erfolgt auf einer Maschine mit
acht CPU-Kernen mit Takt von jeweils 2,4 GHz und 16 GB Arbeitsspeicher.
Aufgrund einiger Firmenregularien kann die Laufzeitmessung der Heuristik nicht auf der gleichen Maschine wie jener des hier entwickelten Programms erfolgen.
Die firmeninterne Maschine weist allerdings sehr ähnliche Hardware-Spezifikationen auf, sodass ein Vergleich dennoch aussagekräftig ist.
Die Laufzeitmessungen erfolgen unter ähnlichen Bedingungen mit minimaler Auslastung durch andere Prozesse.

Ausgewertet wird die Güte der Lösungen nach den Optimierungszielen in Abschnitt \ref{sec:optimierung}.
Zusätzlich kommt als viertes lexikografisches Optimierungsziel die Laufzeit des Programms beziehungsweise der Heuristik hinzu.
Sind alle vier Metriken bei einer Probleminstanz gleich, dann wird die Lösung als gleich gut bewertet.

Die Laufzeitmessungen werden in den folgenden Abschnitten mit verschiedenen oberen Laufzeitgrenzen (Timeouts) versehen.
Gemessen wird die Zeit, welche Programm oder Heuristik für eine endgültige Lösung benötigen.
Ermittelt das Programm also beispielsweise eine Lösung nach $30s$, rechnet aber bis zum Timeout weiter und verbessert diese Lösung nicht,
dann wird die Messung der Laufzeit mit $TO$ (kurz für Timeout) versehen.

Folgend die Erklärung aller Elemente der Vergleiche zur Lösung einer Probleminstanz:
\begin{itemize}
    \item \makebox[2.5cm][l]{Nr:} Identifizierende Nummer der Probleminstanz
    \item \makebox[2.5cm][l]{Name:} Name der Probleminstanz
    \item \makebox[2.5cm][l]{$\lvert C \rvert$:} Anzahl der Komponenten
    \item \makebox[2.5cm][l]{$\lvert C_{Spec} \rvert$:} Anzahl verschiedener Komponenten-Spezifikationen
    \item \makebox[2.5cm][l]{Typ:} SMT (inkr. Verfeinern) oder OMT (lineare Optimierung)
    \item \makebox[2.5cm][l]{Logik:} \texttt{QF\_LIA} oder \texttt{QF\_LRA}
    \item \makebox[2.5cm][l]{Solver:} Externer SMT-Solver
    \item \makebox[2.5cm][l]{$p$:} Minimale Pressen-Anzahl (erklärt in Abschnitt \ref{sec:parallelisierung})
    \item \makebox[2.5cm][l]{Größe:} Summe der Anzahl der Knoten aller Constraints im AST
    \item \makebox[2.5cm][l]{\#$\mathbb{B}$:} Anzahl boolescher Unbekannter im Constraint-System
    \item \makebox[2.5cm][l]{\#$\mathbb{Z}$:} Anzahl numerischer Unbekannter im Constraint-System
    \item \makebox[2.5cm][l]{Zeit in s:} Laufzeit in Sekunden (aufgerundet), \textit{TO} ist Timeout
    \item \makebox[2.5cm][l]{$\lvert C_{Not} \rvert$:} Anzahl unverplanter Komponenten
    \item \makebox[2.5cm][l]{$\lvert P \rvert$:} Anzahl geplanter Pressen
    \item \makebox[2.5cm][l]{Rest in mm:} Gesamtrest in Millimetern
    \item \makebox[2.5cm][l]{Zert.:} \cmark \; falls Lösung zertifiziert optimal ist, \xmark \; sonst
\end{itemize}

\section{Vergleich verschiedener Programm-Konfigurationen}

\section{Vergleich mit Germanedge's Heuristik}

\section{Diskussion}
% Was machen wir schon gut?
% Wo gehts besser? Vorverarbeitung? Nachverarbeitung nicht zertifizierter Lösungen?
% PP-Problem irl mit vielen Verfeinerungen
% Wie günstig kann man die hinzufügen?
