\section{Einleitung}

Die Pressenplanung von Holzbalken aus Bauteilen erfordert eine fein abgestimmte Konfiguration, um die Pressen optimal auszulasten und den Verschnitt der Balken zu minimieren.
Im Rahmen der Bachelorarbeit \textit{Erfüllbarkeit und Optimierung von Pressenplanungsproblemen mit Constraint-Systemen} soll dafür eine prototypische Lösung mithilfe der Methoden der Constraint-Programmierung erarbeitet werden.
Dazu sollen die Constraints der Domäne zunächst mathematisch modelliert und anschließend in einer konkreten Repräsentation des Modells implementiert werden.

Dabei soll ein Programm entwickelt werden, welches aus den textuellen Eingaben der Pressen- und Bauteilspezifikationen ein Constraint-System erzeugt, das anschließend von einem externen Constraint-Solver gelöst wird.
Die Lösung des Solvers soll folglich vom Programm in eine textuelle Ausgabe umgewandelt werden.
Das Programm soll kein grafisches Interface umfassen, sondern lediglich die aktuell existenten Schnittstellen der Ein- und Ausgaben implementieren.
Weiter soll das Programm so strukturiert werden, dass sowohl SAT (Bitblasting) als auch SMT (QF\_LRA, QF\_LIA) zur Erzeugung des Constraint-Systems genutzt werden können.
Als Programmiersprache für das Programm wird Haskell gewählt.

Das Ziel für die prototypische Lösung des Problems der Pressenoptimierung ist die korrekte (optimale) und \textit{schnelle} Lösung bekannter Testfälle.
Sonderfälle, wie beispielsweise Pressen mit Balken unterschiedlicher Länge, sollen vorerst nicht betrachtet werden.

Auf Wunsch von Germanedge kann der Text der Arbeit so gestaltet werden, dass in der Arbeit genannte Anwendungsfälle nur abstrakt beschrieben und Testfälle bereinigt werden.
Die Arbeit soll dabei den gesamten Prozess von Anforderungsanalyse über Entwurf und Realisierung bis hin zum Test dokumentieren.