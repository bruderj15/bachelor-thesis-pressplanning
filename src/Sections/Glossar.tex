\newglossaryentry{OMT}
{
    name=OMT,
    description={Optimization Modulo Theories: SMT mit zusätzlichen symbolischen Optimierungszielen}
}

\newglossaryentry{SMT}
{
    name=SMT,
    description={Satisfiability Module Theories: Erfüllbarkeitsproblem mit Arithmetik}
}

\newglossaryentry{SAT}
{
    name=SAT,
    description={Satisfiability: Aussagenlogisches Erfüllbarkeitsproblem}
}

\newglossaryentry{ADT}
{
    name=ADT,
    description={Algebraischer Datentyp: Zusammengesetzter Datentyp}
}

\newglossaryentry{GADT}
{
    name=GADT,
    description={Generalisierter Algebraischer Datentyp: Algebraischer Datentyp mit möglichem Type-Refinement beim Pattern-Match}
}

\newglossaryentry{SMT-Solver}
{
    name=SMT-Solver,
    description={Ausführbares Programm zum Lösen von SMT-Problemen}
}

\newglossaryentry{SAT-Solver}
{
    name=SAT-Solver,
    description={Ausführbares Programm zum Lösen von SAT-Problemen}
}

\newglossaryentry{Constraint}
{
    name=Constraint,
    description={Allgemein eine Anforderung, hier im Rahmen SMT eine prädikatenlogische Formel}
}

\newglossaryentry{Constraint-System}
{
    name=Constraint-System,
    description={Konjunktion von Constraints}
}

\newglossaryentry{Singleton}
{
    name=Singleton,
    description={Laufzeitzeuge für Übersetzungszeit-Typen}
}

\newglossaryentry{SMTLib}
{
    name=SMTLib,
    description={Sprachstandard zur Definition von SMT-Problemen}
}

\newglossaryentry{SMT-COMP}
{
    name=SMT-COMP,
    description={SMT-Competition: Wettbewerb für SMT-Solver in verschiedenen SMT-Logiken}
}

\newglossaryentry{SMT-Logik}
{
    name=SMT-Logik,
    description={Kombination von Theorien}
}

\newglossaryentry{CDCL(T)}
{
    name=CDCL(T),
    description={Conflict-driven clause learning: Vollständiges Lösungsverfahren für SMT-Probleme bestimmter Logiken}
}

\newglossaryentry{ITE}
{
    name=ITE,
    description={If-Then-Else Ausdruck: Pfadwahl nach Bedingung}
}

\newglossaryentry{Kind}
{
    name=Kind,
    description={Englisch für \textit{Art}: Typ eines Typen}
}

\newglossaryentry{Type-Refinement}
{
    name=Type-Refinement,
    description={Verfeinern eines Typen: Ein allgemeinerer Typ kann durch einen spezifischeren repräsentiert werden}
}

\newglossaryentry{eDSL}
{
    name=eDSL,
    description={Embedded domain-specific language: Eingebettete domänenspezifische Sprache}
}

\newglossaryentry{Incremental-Stack}
{
    name=Incremental-Stack,
    description={Stapel, auf welchem in einem SMT-Solver ebenenweise Variablen angelegt oder Formeln assertiert werden können. Die oberste Ebene des Stapels kann jederzeit entfernt werden.}
}

\newglossaryentry{Incremental Refinement}
{
    name=Incremental Refinement,
    description={Schrittweises Anpassen einer Schranke mit der Suche nach einem Optimum}
}
