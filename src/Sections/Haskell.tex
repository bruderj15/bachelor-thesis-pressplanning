\chapter{Haskell und Hasmtlib}
\label{chapter:haskell}
In diesem Kapitel werden einige, für das weitere Verständnis der Arbeit notwendige, Grundlagen der funktionalen Programmiersprache Haskell erklärt.
Danach wird der Kern der Funktionsweise der Haskell-Bibliothek \texttt{Hasmtlib} \cite{hasmtlib} beschrieben.
Ferner werden einige geeignete Erweiterungen dieser diskutiert.

\section{Funktionale Programmierung mit Haskell}
\label{sec:haskell}
Haskell ist eine funktionale Programmiersprache, welche sich vor allem durch statisch polymorphe Typisierung und Bedarfsauswertung auszeichnet \cite{haskell2010report}.
Besonders das ausdrucksstarke Typsystem macht Haskell zu einem mächtigen Werkzeug in der Software-Entwicklung.
Haskell unterscheidet dabei zwischen Typen zur Übersetzungszeit und Werten zur Laufzeit eines Programms \cite{singletons}.
Diese strenge Trennung wird als Phasentrennung bezeichnet.
Betrachten wir folgenden algebraischen Datentypen (kurz \gls{ADT}) einfach gelinkter Listen:
\begin{listing}[H]
    \inputminted[linenos=true]{haskell}{Code/Implementierung/Haskell/List.hs}
    \caption{Algebratischer Datetyp einfach gelinkter Listen}
    \label{listing:linkedlist}
\end{listing}
Dieser Listen-Typ ist polymorph in den Elementen der Liste, sodass \texttt{a} mit verschiedenen Typen belegt werden kann.
In Zeile 3 von Listing \ref{listing:linkedlist} wird das \texttt{a} zum Beispiel mit \texttt{Int} belegt.
Damit ist \texttt{exampleList} ein Wert vom Typ \texttt{List Int}, also in Haskell-Syntax: \texttt{exampleList :: List Int}.

Diese \textit{ist-Wert-von-Typ-Beziehung} kann auch auf die Typ-Ebene zur Übersetzungszeit gehoben werden.
Die höhere Beziehung lautet dann: \textit{ist-Typ-von-Art-Beziehung}.
Somit ist \texttt{List Int} vom \textit{\gls{Kind}} (deutsch Art) \texttt{Type}, also in Haskell-Syntax: \texttt{List Int :: Type}.
Ein \textit{Kind} für einen Typen ist also das, was ein Typ für einen Wert ist.

Das gilt auch für Funktionen.
Die Funktion \texttt{append} nimmt zwei Listen und produziert eine Liste, also \texttt{append :: List a -> List a -> List a}.
Der polymorphe Typ \texttt{List a} ist auch eine Funktion - aber auf Typ-Ebene.
Er nimmt einen Typen und produziert einen Typen, also \texttt{List a :: Type -> Type}.
Die Wahl des \texttt{a} kann durch Typ-Applikationen zur Übersetzungszeit folgend gewählt werden:
\begin{listing}[H]
    \inputminted[linenos=true]{haskell}{Code/Implementierung/Haskell/TypeApplication.hs}
    \caption{Typ-Applikation polymorpher Funktionen}
    \label{listing:typeapplication}
\end{listing}
Intern werden dazu Type-Operatoren (\texttt{\raisebox{-0.6ex}{\textasciitilde}}) verwendet.
In Listing \ref{listing:typeapplication} wird das \texttt{a} mit \texttt{Float} belegt, also gilt: \texttt{a \raisebox{-0.6ex}{\textasciitilde} Float}.

Mit diesem Wissen können nun Klassen betrachtet werden.
Abstraktion über Listing \ref{listing:linkedlist} liefert die Spezifikation von Halbgruppen:
\begin{listing}[H]
    \inputminted[linenos=true]{haskell}{Code/Implementierung/Haskell/Semigroup.hs}
    \caption{Typ-Klasse für Halbgruppen}
    \label{listing:semigroup}
\end{listing}
Die Klasse \texttt{Semigroup} spezifiziert die Funktion \texttt{(<>)} und der Typ \texttt{List a} implementiert diese Spezifikation.

Bisher wurde die Implementierung von \texttt{append} in Listing \ref{listing:linkedlist} noch nicht betrachtet.
Die Implementierung unterscheidet sich dabei je nach Konstruktor des ADTs \texttt{List a}.
Das Aufbrechen der Konstruktoren wird auch \textit{Pattern-Matching} genannt \cite{haskell2010report}.
Eine Verallgemeinerung von ADTs stellen generalisierte algebraische Datentypen (kurz \gls{GADT}) dar.
Dort kann der Pattern-Match auf einen Wert (Konstruktor) zur Laufzeit zusätzlich Informationen zum Typen des Werts zur Übersetzungszeit liefern.
Da dadurch Typen verfeinert werden können, wird diese Möglichkeit auch \textit{\gls{Type-Refinement}} genannt \cite{gadts}.

Die bisher betrachteten Funktionen sind alle pur - also mathematische Funktionen.
Um zusätzlich Nebenwirkungen zu modellieren, nutzt Haskell das Konzept der Monade \cite{haskell2010report}.
Da die Herleitung jener für diesen Rahmen zu komplex ist, reicht es für den Leser, zu verstehen, dass
monadische Funktionen Berechnungen mit Kontext sind.
Dieser Kontext ist bereits am Typ zu erkennen, sodass \texttt{appendPrint} beispielsweise den Typ \texttt{List a -> List a -> IO (List a)} hätte.

\section{Hasmtlib - Haskell-Bibliothek für SMTLib}
\label{sec:hasmtlib}
Die Haskell-Bibliothek \texttt{Hasmtlib} \cite{hasmtlib} ist eine eingebettete domänenspezifische Sprache
(englisch \textit{embedded domain-specific language} - kurz \textit{eDSL}) für den SMTLib-Standard Version 2.6 \cite{smtlib} in Haskell.
Eine \gls{eDSL} ist eine, in einer Gastsprache eingebettete, anwendungsspezifische Sprache, welche auf Grundkonstrukten der Gastsprache aufbaut und somit den
Implementierungsaufwand der eingebetteten Sprache erheblich reduziert \cite{eDSL}.

Haskell wird aufgrund seines starken Typsystems oft als Gastsprache für solche eingebetteten Sprachen verwendet \cite{eDSL2}.
Listing \ref{listing:hasmtlibexprgadt} zeigt beispielsweise, wie \texttt{Hasmtlib} SMTLib-Ausdrücke als abstrakten Syntaxbaum mit einem GADT repräsentiert.

\begin{listing}[H]
    \inputminted[linenos=true]{haskell}{Code/Implementierung/Hasmtlib/ExprGADT.hs}
    \caption{Repräsentation von SMTLib-Ausdrücken mithilfe eines GADTs in \texttt{Hasmtlib} \cite{hasmtlibExpr}}
    \label{listing:hasmtlibexprgadt}
\end{listing}

Dabei ist der GADT \texttt{Expr t} polymorph in der \texttt{SMTSort}, sodass dessen Kind \texttt{SMTSort -> Type} ist.
Da Werte des Datentyps \texttt{SMTSort} hier als Typen verwendet werden, nennt man \texttt{SMTSort} auch \textit{Data-Kind} oder \textit{Promoted-Type} \cite{singletons}.
Besonders relevant dafür ist das in \ref{sec:haskell} erklärte Type-Refinement.
So weist beispielsweise ein Pattern-Match auf den Konstruktor \texttt{Sqrt} bereits zur Übersetzungszeit nach, dass \texttt{t \raisebox{-0.6ex}{\textasciitilde} RealSort} gilt.
Bei anderen Konstruktoren wie \texttt{And} kann die Belegung des \texttt{t} in \texttt{Expr t} mithilfe von Singletons ermittelt werden.
Ein \textit{\gls{Singleton}} ist ein Laufzeit-Zeuge (Wert) für genau einen Übersetzungszeit-Typen \cite{singletons}.
Diese Abhängigkeit ist aufgrund Haskell's strenger Phasentrennung notwendig und ermöglicht eine dependent-ähnliche Kodierung
wie in den Dependently-Typed Programmiersprachen Idris \cite{idris} und Agda \cite{agda}.

Jeder Typ vom Kind \texttt{SMTSort} erhält also einen Singleton (Wert).
Wie in Listing \ref{listing:hasmtlibssmtsort} zu sehen, stellt \texttt{Hasmtlib} den Singleton-Typen \texttt{SSMTSort} ebenfalls als GADT dar.

\begin{listing}[H]
    \inputminted[linenos=true]{haskell}{Code/Implementierung/Hasmtlib/SSMTSort.hs}
    \caption{Repräsentation des Singleton-Typs \texttt{SSMTSort} in \texttt{Hasmtlib} \cite{hasmtlibSSMTSort}}
    \label{listing:hasmtlibssmtsort}
\end{listing}

Die Verbindung zwischen Typ und Wert wird mit der in Listing \ref{listing:hasmtlibknownsmtsort} dargestellten Typ-Klasse \texttt{KnownSMTSort} erreicht.

\begin{listing}[H]
    \inputminted[linenos=true]{haskell}{Code/Implementierung/Hasmtlib/KnownSMTSort.hs}
    \caption{Typ-Klasse zur Verbindung von Typ \texttt{SMTSort} zu Wert \texttt{SSMTSort} in \texttt{Hasmtlib} \cite{hasmtlibKnownSMTSort}}
    \label{listing:hasmtlibknownsmtsort}
\end{listing}

Mit den genannten Mechanismen garantiert \texttt{Hasmtlib} die ausschließliche Erzeugung von sauber typisierten (englisch \textit{well-typed}) Ausdrücken \cite{gadts}.
Diese werden anschließend von der Bibliothek in SMTLib-Syntax übersetzt und an einen SMT-Solver gegeben.
Dessen Antwort parst und übersetzt \texttt{Hasmtlib} dann zurück in die entsprechenden Haskell-Typen.
Folgend ein kleines Programm zur Erläuterung:

\begin{listing}[H]
    \inputminted[linenos=true]{haskell}{Code/Implementierung/Hasmtlib/BeispielSimple.hs}
    \caption{Beispielhafte Verwendung von Hasmtlib}
    \label{listing:hasmtlibsimplexample}
\end{listing}

Das Beispiel in Listing \ref{listing:hasmtlibsimplexample} demonstriert mit der Funktion \texttt{barLength} außerdem die pure Formelkonstruktion in \texttt{Hasmtlib}.
Lediglich Variablenerzeugung (Zeile 22) und Formel-Assertion (Zeile 24) sind monadisch.
Dadurch lassen sich viele Funktionalitäten der Gastsprache wiederverwenden, zum Beispiel die Funktion \texttt{sum} aus \texttt{Data.Foldable}:
\texttt{sum :: (Foldable t, Num a) => t a -> a}.

\subsection{Ergänzung um Anzahl-Constraints}
\label{subsec:implCardinality}
\texttt{Hasmtlib} unterstützte bis zum Zeitpunkt dieser Arbeit keine Anzahl-Constraints.
Daher werden diese im Rahmen dieser Arbeit der Bibliothek hinzugefügt.
Die Implementierung erfolgt analog zu den in \ref{sec:smtcardinality} präsentierten Kodierungen und ist in
Anhang \ref{appendix:hasmtlib} unter Listing \ref{listing:hasmtlibcardinality} zu finden.

\subsection{Ergänzung um Incremental Refinement}
\label{subsec:incrementalrefinement}
\texttt{Hasmtlib} kann aufbauend auf SMT-Problemen auch OMT-Probleme kodieren.
Da dies allerdings nur wenige SMT-Solver wie \texttt{Z3} direkt unterstützen, verwendet man häufig sogenannte Optimierungsschleifen \cite{nuz3}.
Dabei werden iterativ bessere Lösungen gesucht, indem obere Schranken für Maxima oder untere Schranken für Minima verfeinert werden.
SMT-Solver haben einen \gls{Incremental-Stack}, welcher die Wiederverwendung bereits definierter Probleme und auch erlernter Lemmas ermöglicht,
sodass \textit{\gls{Incremental Refinement}} wesentlich effizienter als das Starten eines neuen Solver-Prozesses für jede Verfeinerung ist \cite{incremental}.

Im Rahmen dieser Arbeit wird \texttt{Hasmtlib} daher um eine solche Optimierungsschleife erweitert.
Deren Implementierung ist im Anhang \ref{appendix:hasmtlib} in Listing \ref{listing:incrementalrefinement} zu finden.
Wie dort zu erkennen ist, wurde außerdem eine Schrittweite für das Anpassen der verfeinerten Schranken hinzugefügt.
Um nicht über die optimale Lösung zu schreiten, ruft Zeile 37 die Optimierungsschleife nochmals - aber ohne Schrittweite - auf.
Die letzte bekannte optimale Schranke liegt dabei noch auf dem Incremental-Stack, sodass der rekursive Aufruf mit dieser als neue Schranke startet.
Das garantiert das Finden der optimalen Lösung.
