\chapter{Implementierung}
\label{chapter:implementierung}
In diesem Kapitel wird gezeigt, wie das Problem der Pressenplanung mit Haskell und \texttt{Hasmtlib} in SMTLib kodiert werden kann.
Dazu werden zunächst die notwendigen Datentypen und die Sorten-Polymorphie der Kodierung erklärt, bevor einige konkrete Repräsentationen der Constraints aus
Abschnitt \ref{sec:modellierung} mithilfe von Quellcode-Ausschnitten dargestellt werden.
Danach werden Ein- und Ausgabe kurz beschrieben.
Anschließend wird dargelegt, wie die Implementierung mit bestimmten Parametern konfiguriert werden kann und wie diese die Laufzeit des Solvers beeinflussen.
Am Ende wird eine Möglichkeit zur Parallelisierung des Programms demonstriert.

% TODO: Hier ein Flussdiagramm einfügen, das den gesamten Ablauf zeigt? Oder am Ende des Kapitels?

\section{Datentypen und Unbekannte}
Analog zur Problemdomäne aus Kapitel \ref{chapter:problem} sind hier Bauteile (Components), Schichten (Layers) und Pressen (Presses)
die Entitäten von Bedeutung.
Deren Repräsentation erfolgt jeweils durch Record-Datenstrukturen.
Bei Pressen wird zusätzlich in statische und symbolische Pressen unterschieden.
Statische Pressen (\texttt{StaticPress}) beschreiben die konstanten Eigenschaften von Pressen wie beispielsweise deren maximale Höhe.
Symbolische Pressen (\texttt{Press}) haben die Konstanten einer statischen Presse und beschreiben weiter symbolische Eigenschaften wie die geplante Länge in einem Pressenplan.
Diese Unterscheidung ist für die Trennung von Eingabe (Einlesen der Pressenspezifikationen) und Verarbeitung der Pressen im Constraint-System sinnvoll.
Folgend deren Haskell-Repräsentation:

\begin{listing}[H]
    \inputminted[linenos=true]{haskell}{Code/Implementierung/Datatypes.hs}
    \caption{Haskell-Datentypen der Domäne Pressenplanung}
    \label{listing:datatypes}
\end{listing}

Der Ausdruck \texttt{\$(makeLenses ''DATATYPE)} taucht dabei mehrmals auf.
Dieser erzeugt \textit{Lenses} für die Datentypen.
Da im Weiteren Lenses häufig in der Implementierung verwendet werden, wird dem Leser empfohlen, folgende Quelle bei Fragen zu Lenses zu konsultieren: \cite{lenses}.

Ferner ist auffällig, dass alle Typen polymorph sind, sodass \texttt{Layer a b} beispielsweise zwei Typen \texttt{a} und \texttt{b} nimmt.
Das \texttt{a} repräsentiert einen Zahlentypen und das \texttt{b} einen Bool-Typen.
Das vereinfacht später die Kodierung und Dekodierung der symbolischen Werte.
So könnte die symbolische Schicht mit \texttt{type SLayer t = Layer (Expr t) (Expr BoolSort)}
und die konkrete Schicht mit \texttt{type CLayer t = Layer (HaskellType t) Bool} dargestellt werden.

Mit diesen Grundtypen kann nun der Typ \texttt{Pressenplan} definiert werden:

\begin{listing}[H]
    \inputminted[linenos=true]{haskell}{Code/Implementierung/Datatypes2.hs}
    \caption{Haskell-Datentyp des Pressenplans}
    \label{listing:datatypes2}
\end{listing}

Dieser enthält alle symbolischen Komponenten, Schichten, Pressen und auch die Relation \textit{InLayer} aus Abschnitt \ref{sec:modellierung}.
Als Index für diese wird das Tupel \texttt{(cId, lId)} aus Komponenten-ID und Schicht-ID verwendet.

Die Erzeugung eines symbolischen Pressenplans ist monadisch (\texttt{setupPressPlan :: MonadSMT s m => ... -> m (PressPlan t)}), da dort alle notwendigen Variablen angelegt werden.
Identisch zur Variablenmenge $\mathbb{X}$ in Abschnitt \ref{sec:modellierung} sind das:

\begin{itemize}
    \item \texttt{\_pressLength} und \texttt{\_pressHeight} für jede Presse,
    \item \texttt{\_layerLength}, \texttt{\_layerHeight}, \texttt{\_waste} und \texttt{\_layerEmpty} für jede Schicht und
    \item ein symbolischer Wahrheitswert für jedes mögliche Element der Relation \texttt{InLayer}
\end{itemize}

Bei weiterer Betrachtung des Typs \texttt{Pressenplan} fällt auf, dass Komponenten, Schichten und Pressen im Typ Liste vorliegen.
Der Listentyp ist der von Haskell, also konkret in Haskell und nicht symbolisch im SMT-Problem.
Das erklärt die Nutzung der Liste für die Komponenten, da diese eine Eingabe sind, welche konkret darstellbar ist.
Aber wie ist der Listentyp für Schichten und Pressen zu erklären?

Das Modell in Abschnitt \ref{sec:modellierung} all-quantifiziert Schichten und Pressen.
Das wäre analog auch in SMT möglich.
Im Allgemeinen werden solche Probleme aber grundinstanziiert,
um die Verwendung ineffizient (oder gar nicht) entscheidbarer Quantoren-Logiken zu vermeiden \cite{smtquantifiersarehard, smtquantifiersarehard2}.
Für jede Presse wird dabei im Voraus die höchstmögliche Anzahl an Schichten basierend auf der Eingabe an Komponenten ermittelt, genauer in \ref{sec:vorverarbeitung}.
Da Schichten leer sein können haben sie das Attribut \texttt{\_layerEmpty :: b} (Listing \ref{listing:datatypes}, Zeile 13).
Identisch könnten Pressenpläne leere Pressen enthalten.
Weiter könnte eine obere Schranke für die Anzahl an Pressen basierend auf der Eingabe an Komponenten ermittelt werden.
Allerdings bedeutet jede weitere Presse weitere Unbekannte und weitere Constraints, sodass die Komplexität des Problems erhöht wird,
was letztendlich zu einer ineffizienteren Lösung des Solvers führt.
Daher wird jeder Auftrag mehrmals kodiert - mit jeweils unterschiedlich fester Anzahl an Pressen.
Diese werden dann als separate Probleminstanzen auf verschiedene Prozessor-Kerne aufgeteilt, mehr dazu in Abschnitt \ref{parallelisierung}.

\section{Vorverarbeitung}
\label{sec:vorverarbeitung}

\section{Sorten-Polymorphie}
Wie in Kapitel \ref{chapter:smt} gezeigt bietet SMTLib verschiedene Logiken zur Kodierung von SMT-Problemen an.
Nicht immer ist dabei vorher klar, in welcher Logik ein Problem am effizientesten für Solver kodiert werden kann.
Das gilt auch für das Pressenplanungsproblem mit den Kandidaten \texttt{QF\_LIA}, \texttt{QF\_LRA} und \texttt{QF\_BV}.
\texttt{Hasmtlibs} Ausdrucks-Typ \texttt{Expr t} kann dafür genutzt werden, um das Problem polymorph in der Sorte zu implementieren.
Betrachten wir folgende Funktion, die für einen Pressenplan und dessen Konfiguration alle Constraints für jenen erstellt:

\begin{listing}[H]
    \inputminted[linenos=true]{haskell}{Code/Implementierung/SortPoly.hs}
    \caption{Sorten-Polymorphie bei der Problemdeklaration}
    \label{listing:implsortpoly}
\end{listing}

Der Aufrufer wählt die Sorte der Kodierung dann per Typ-Applikation und setzt die Logik mithilfe der Singletons:

\begin{listing}[H]
    \inputminted[linenos=true]{haskell}{Code/Implementierung/SortPolySetLogic.hs}
    \caption{Auswahl der Logik nach Sorte}
    \label{listing:implsortpolylogic}
\end{listing}

Das ermöglicht die Wahl der Kodierung für den Aufrufer der Schnittstelle und verhindert somit Redundanz in der Implementierung.
Insbesondere begünstigt das die einfache Testung verschiedener Kombinationen von Solvern und Kodierungen in verschiedenen Logiken wie später bei der Auswertung in Kapitel \ref{chapter:auswertung}.

Für die Sorte \texttt{Real} werden alle Längenangaben in der Einheit Meter kodiert, sodass dabei offensichtlich keine Lösungen aufgrund von unzureichender Genauigkeit verloren gehen kann.
Bei den Sorten \texttt{Int} und \texttt{BitVec} wird hingegen die Einheit Millimeter verwendet.
Da alle Eingaben durch natürliche Zahlen in Millimetern dargestellt werden und im Constraint-System keine Division erfolgt, geht auch hier keine Lösung aufgrund von Ungenauigkeit verloren.
Für \texttt{BitVec} reichen 15 Bit bei vorzeichenloser Interpretation aus, da die größten Werte die Längen der Pressen (höchstens 25000 mm) immer kleiner als $2^{15} = 32768$ sind.
Wie in \cite{bitblasting} beschrieben, skalieren Bitvektor-Kodierungen allerdings schlecht in Bezug auf die Bitbreite.
Bei deren Reduktion auf äquivalente SAT-Kodierungen entstehen dabei besonders für arithmetische Operationen große Formeln.
Obwohl für den Anwendungsfall Pressenplanung hier nur Bitvektoren der Länge 15 verwendet werden, führt die Verwendung von zahlreichen arithmetischen Operationen
vermutlich zu ineffizienten Kodierungen.
Diese Vermutung konnte durch - in dieser Arbeit nicht dokumentierte - Testfälle bestätigt werden.
Daher werden im folgenden der Arbeit Bitvektoren nicht weiter für die Kodierung in Betracht gezogen.

\section{Kodierung der Constraints}
\label{sec:kodierungconstraints}
Die Constraints aus Abschnitt \ref{sec:modellierung} lassen sich dank \texttt{Hasmtlib} analog in Haskell formulieren.
Betrachten wir die Kodierung weniger Constraints wie \ref{constraint:pressHeight} und \ref{constraint:pressLength} aus Abschnitt \ref{sec:modellierung}:

\begin{listing}[H]
    \inputminted[linenos=true]{haskell}{Code/Implementierung/ConstraintPressenDimensionen.hs}
    \caption{Haskell-Kodierung der Constraints \ref{constraint:pressHeight} und \ref{constraint:pressLength} aus Abschnitt \ref{sec:modellierung}}
    \label{listing:constraintimplpressdim}
\end{listing}

Hier ist darauf hinzuweisen, dass die Funktion \texttt{all} in Zeile 2 nicht all-quantifiziert, sondern über allen Pressen des Problems grundinstanziiert.
Dabei entsteht also aus der Konjunktion der Anforderungen an jede einzelne Presse eine Formel, welche die Anforderung für alle Pressen kodiert.
Das bestätigt auch der Rückgabetyp \texttt{Expr BoolSort} der Funktion \texttt{pressBounds}.

Betrachten wir weiter folgend das Constraint \ref{constraint:eoLayer}:

\begin{listing}[H]
    \inputminted[linenos=true]{haskell}{Code/Implementierung/ConstraintCompInAmoLayer.hs}
    \caption{Haskell-Kodierung des Constraints \ref{constraint:eoLayer}}
    \label{listing:constraintimplamolayer}
\end{listing}

Es beschreibt, dass jede Komponente genau einer Schicht zugeordnet wird.
Die Funktion \texttt{exactly} in Zeile 3 ist dabei die in Abschnitt \ref{sec:smtcardinality} erklärte und in Abschnitt \ref{subsec:implCardinality} implementierte
Exactly-One-Kodierung.

Zusätzlich zu den Constraints des Modells in Abschnitt \ref{sec:modellierung} muss zusätzlich folgendes Constraint hinzugefügt werden.

\begin{listing}[H]
    \inputminted[linenos=true]{haskell}{Code/Implementierung/ConstraintCompInAmoLayer.hs}
    \caption{Haskell-Kodierung des Constraints \ref{constraint:eoLayer}}
    \label{listing:constraintpositivewaste}
\end{listing}

Im Modell sind die Unbekannten Elemente der Menge $\mathbb{N}_0$.
In SMTLib repräsentieren die Sorten \texttt{Int} und \texttt{Real} jedoch die Mengen $\mathbb{Z}$ und $\mathbb{R}$.
Daher muss sichergestellt werden, dass alle Unbekannten nicht-negativ sind.
Sowohl für Pressenlänge und Pressenhöhe als auch Schichtlänge und Schichthöhe ergibt sich das bereits aus den Constraints in Abschnitt \ref{sec:modellierung}.
Wie aus Listing \ref{listing:constraintpositivewaste} zu schließen allerdings nicht für den Schichtrest, sodass obiges Constraint hinzugefügt werden muss.

\section{Parallelisierung und Optimierung}
%Incremental Refinement/OMT, Schrittweite
% TODO: Das hier könnte man viel schöner als formale Folgerungskette schreiben
In Abschnitt \ref{sec:optimierung} wurden drei Optimierungsziele genannt.
Optimierungsziel 1, die Minimierung unverplanter Komponenten ist wie dort bereits beschrieben immer durch einen Pressenplan mit genau null übrigen Komponenten erfüllbar.
Optimierungsziel 2 ist die Minimierung der Anzahl geplanter Pressen.
Da erfahrungsgemäß selbst bei den größten Aufträgen höchstens zehn Pressen pro Auftrag geplant werden, gilt also $1 \leq \lvert P \rvert \leq 10$.
Wird ein Auftrag also zehnmal mit jeweils unterschiedlicher Pressen-Anzahl kodiert, dann enthält eine jener Kodierungen die optimale Belegung.
Optimal ist jene Belegung in entsprechender Kodierung genau dann, wenn der Gesamtrest unter der Belegung minimal ist (Optimierungsziel 3).
Das kleinste Optimum des Gesamtrests liegt bestenfalls bei null.
Durch das Hinzufügen des Constraints $\forall p \in P: 0 = \sum_{l \in \text{layers}(P)} \text{waste}(l)$ zur Kodierung mit der minimalen Pressen-Anzahl
kann dieses Optimum forciert werden.
Ist das Problem nach Hinzufügen des Constraints noch immer erfüllbar, dann ist jede erfüllende Belegung des Problems optimal.
Ein solches zusätzliches Constraint schränkt den Lösungsraum des Problems erheblich ein, sodass die Lösung des Problems durch den Solver wesentlich effizienter wird.
Hier nicht dokumentierte Testfälle bestätigen diese Annahme.

Allerdings kann nicht garantiert werden, dass es eine Lösung ohne Rest gibt.
Daher wird ein Auftrag 20 Mal kodiert: Auf zehn verschiedene Pressen-Anzahlen in jeweils zwei Variationen:
Mit dem extra Constraint, welches den Gesamtrest auf null fixiert und ohne dieses.
Diese 20 verschiedenen Kodierungen eines Auftrages werden bestenfalls auf 20 verschiedene Kerne einer CPU (oder mehrere) verteilt.
Die Maschine des Kunden, welcher Germanedge mit dem Pressenplanungsproblem beauftragt hat, verfügt allerdings nur über eine CPU mit acht Kernen.
Daher wird ein Auftrag mit vier verschiedenen Pressen-Anzahlen kodiert, sodass insgesamt acht Kodierungen des Auftrags entstehen.
Ein Eingabeparameter $p \in \mathbb{N}^+$ des Programms bestimmt dabei die Mindestanzahl an Pressen.
Für ein bestimmtes $p$ wird ein Auftrag dann für $p$, $p+1$, $p+2$ und $p+3$ kodiert.
Dabei könnte offensichtlich die optimale Lösung mit Pressen-Anzahl $p\prime$ verloren gehen, da $p\prime < p \oplus p\prime > p + 3$ gelten könnte.
Ähnlich wie bei der Ermittlung der maximalen Schichtanzahl pro Presse in Abschnitt \ref{sec:vorverarbeitung} könnte auch hier ein
(vermutlich linearer) Algorithmus entwickelt werden, der die Mindestanzahl an Pressen ermittelt.
Da in dieser Arbeit allerdings lediglich ein Prototyp und kein vollständiges Produkt zur optimalen Lösung des Pressenplanungsproblems erarbeitet werden soll,
wird als Eingabeparameter ein niedriges $p$ basierend auf Germanedge's interner Heuristik gewählt.
Dennoch kann auch damit teilweise die Optimalität einer Lösung nachgewiesen werden.
Zum Beispiel, wenn die Kodierung eines Auftrages mit $p$ unerfüllbar und mit $p+1$ erfüllbar ist.
Dann ist $p+1$ die minimale Pressen-Anzahl.

Symbolisch wird also nur der Gesamtrest optimiert (Optimierungsziel 3).
Das geschieht entweder durch inkrementelles Verfeinern (SMT) mit der Implementierung aus Abschnitt \ref{subsec:incrementalrefinement} oder durch
die linearen Optimierungsverfahren von Z3 (OMT).
Um bei ersterer realistische Verbesserungen der Lösung zu gewährleisten, wird als Schrittweite ein Millimeter der Kodierung verwendet.
Also $0.001$ Einheiten (Meter) bei Kodierung in der Sorte \texttt{Real} und $1$ Einheit (Millimeter) bei der Kodierung in der Sorte \texttt{Int}.
Der Aufruf der Optimierungsfunktion aus Abschnitt \ref{subsec:incrementalrefinement} sieht somit folgenderweise aus:

\begin{listing}[H]
    \inputminted[linenos=true]{haskell}{Code/Implementierung/OptimizationTotalWaste.hs}
    \caption{Aufruf der Optimierungsfunktion \texttt{solveMinimized} für den Gesamtrest}
    \label{listing:solveminimized}
\end{listing}

\section{Benutzerschnittstelle}
Germanedge's bisherige heuristische Lösung spezifiziert Ein- und Ausgabeformat als CSV.
Diese Spezifikation wird hier ebenfalls implementiert.
Dazu wird die Haskell-Bibliothek \texttt{cassava} \cite{cassava} verwendet.
Folgend das Lesen eines Auftrages und das Schreiben der Lösungen für diesen:

\begin{listing}[H]
    \inputminted[linenos=true]{haskell}{Code/Implementierung/CassavaImport.hs}
    \caption{CSV-Verarbeitung mithilfe der Haskell-Bibliothek \texttt{cassava}}
    \label{listing:cassavaimport}
\end{listing}

Der Aufruf des Programms erfolgt über die Kommandozeile mit wenigen Konfigurationsparametern.
Konfiguriert werden können:

\begin{itemize}
    \item Timeout: Maximale Laufzeit des Programms
    \item Art der Kodierung: \texttt{Int} mit \texttt{QF\_LIA} (\texttt{OpenSMT}), \texttt{Real} mit \texttt{QF\_LRA} (\texttt{OpenSMT}) oder \texttt{BitVec 15 (Unsigned)} mit \texttt{QF\_BV} (\texttt{Bitwuzla})
    \item Pressen-Anzahl $p$: Kodiert wird dann mit $p$, $p+1$, $p+2$ und $p+3$
    \item Constraints: Maximaler Gesamtrest, maximaler Rest pro Schicht. Hier könnten optimale Lösungen verloren gehen. Der Kunde wünscht sich diese Parameter dennoch.
    \item Ausgabe-Verzeichnis: Verzeichnis, in welches die Lösungen der verschiedenen Pressen ausgegeben werden
\end{itemize}
