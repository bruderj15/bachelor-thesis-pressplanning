\chapter{Implementierung}
\label{chapter:implementierung}
In diesem Kapitel wird gezeigt, wie das Problem der Pressenplanung mit Haskell und \texttt{Hasmtlib} in SMTLib kodiert werden kann.
Dazu werden zunächst die notwendigen Datentypen und die Sorten-Polymorphie der Kodierung erklärt, bevor einige Implementierungen der Constraints aus
Kapitel \ref{chapter:problem} Abschnitt \ref{sec:modellierung} mithilfe von Quellcode-Ausschnitten dargestellt werden.
Danach werden Ein- und Ausgabe kurz beschrieben.
Anschließend wird dargelegt, wie die Implementierung in bestimmten Parametern konfiguriert werden kann.
Am Ende wird eine Möglichkeit zur Parallelisierung des Programms demonstriert.

\section{Datentypen}

\section{Sorten-polymorphe Kodierung}

\section{Kodierung der Constraints}
\label{sec:kodierungconstraints}

\section{IO}

\section{Konfiguration}
Hier maxLayers erklären.
Und Config-Parameter.

\section{Parallelisierung}