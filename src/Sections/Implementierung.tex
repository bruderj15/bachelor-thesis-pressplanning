\section{Implementierung in Haskell}
Erwähnen, dass Germanedge bereits Lösung hat.
Aber: Heuristik - (optimale) Lösungen gehen möglicherweise verloren.
Hier: Neuer Ansatz mit Garantie Vollständigkeit und Optimalität.
Was ist Haskell und warum Haskell?
Was ist Hasmtlib?

\subsection{Funktionale Programmierung mit Haskell}

\subsection{Hasmtlib - Haskell-Bibliothek für SMTLib}
Hier müssen besonders der polymorphe Formelbaum (Expr-GADT) und die dependent-like Singletons (SSMTSort) erklärt werden.
Wird später gebraucht um zu verstehen, wie Pressenplanungsprobleme flexibel in verschiedenen Logiken kodiert werden (Programm polymorph in SMTSort).

\subsubsection{Ergänzung um Anzahl-Constraints}

\subsubsection{Ergänzung um Incremental Refinement}

\subsubsection{Ergänzung um Observable-Sharing}
Sprengt das den Rahmen?
Hier müsste vorher viel erklärt werden:
Repräsentation von Bäumen im Speicher in Haskell (DAG),
Auszug Kategorientheorie - Morphismen (hier Paramorphismus),
Vergleich zu Tseitin sinnvoll, aber bisher SAT gänzlich unerwähnt.
Dann auch die Frage: Warum der Aufwand?
Bringt das überhaupt was für den Anwendungsfall Pressenplanung?
Wenn nein: Gehört das hier dann überhaupt rein?

\subsection{Kodierung des Pressenplanungsproblems}
Zwei Modelle erschweren hier möglicherweise Verständnis für den Leser.
Für ein Modell entscheiden: Relationales Modell begründend wegwerfen, auf Laufzeitmessung verweisen.
Ebenso Unterscheidung Incremental Refinement/OMT, OMT wegwerfen, weil Laufzeit.
Geeignete Programmauszüge: Polymorphe Kodierung in Sorte (damit Logik), vereinzelt Datentypen und Constraints analog zu Modellierung.