\chapter{Satisfiability Modulo Theories}
\label{chapter:smt}

In diesem Kapitel wird zuerst das Erfüllbarkeitsproblem der Satisfiability Module Theories - kurz SMT - vorgestellt.
Danach wird auf die Kernelemente des Sprachstandards SMTLib Version 2.6 \cite{smtlib} eingegangen.
Dieser ermöglicht die Kodierung von SMT-Problemen für SMT-Solver \cite{smt}.
Ein SMT-Solver - kurz Solver - ist ein Programm, welches für ein kodiertes Problem die Erfüllbarkeit dieses und für den
Fall \textit{erfüllbar} (englisch \textit{satisfiable} - kurz \textit{sat}) ein Modell für jenes bestimmen kann.
Einige bekannte Solver sind zum Beispiel Z3 \cite{z3}, CVC5 \cite{cvc5}, Yices \cite{yices} und OpenSMT \cite{opensmt}.
Diese und viele andere nehmen jährlich an dem internationalen Wettbewerb SMT-COMP \cite{smtcomp} für SMT-Solver teil,
wo sie sich in verschiedenen Disziplinen miteinander messen.
Am Ende dieses Kapitels wird zudem betrachtet, wie SMT-Probleme nach Zielfunktionen optimiert werden können.

\section{Einführung in das Erfüllbarkeitsproblem SMT}
Logiken, Sorten erklären.
Welche Logiken hier relevant?
Wie lösen die Solver die Probleme der Logiken?
Wie performant lösen die Solver diese - Komplexität?
Aktuelle SMT-Comp - beste Solver für relevante Kategorien nennen.

\section{Sprachstandard SMTLib Version 2.6}
Standard Version 2.6: \cite{smtlib}

\section{SMT-Kodierung von Anzahl-Constraints}
Hier folgende Paper: \cite{suelflow} und \cite{kovasznai}

\section{Optimierung - Optimization Modulo Theories}
nuZ3 Paper hier sehr hilfreich \cite{nuz3}