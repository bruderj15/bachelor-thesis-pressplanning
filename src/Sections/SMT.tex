\chapter{Satisfiability Modulo Theories}
\label{chapter:smt}

In diesem Kapitel wird zuerst das Erfüllbarkeitsproblem der Satisfiability Module Theories - kurz SMT - vorgestellt.
Dabei wird auch kurz auf die Kernelemente des Sprachstandards SMTLib Version 2.6 \cite{smtlib} eingegangen.
Dieser ermöglicht die Kodierung von SMT-Problemen für SMT-Solver \cite{smt}.
Ein SMT-Solver - kurz Solver - ist ein Programm, welches für ein kodiertes Problem die Erfüllbarkeit dieses und für den
Fall \textit{erfüllbar} (englisch \textit{satisfiable} - kurz \textit{sat}) ein Modell für jenes bestimmen kann.
Einige bekannte Solver sind zum Beispiel Z3 \cite{z3}, CVC5 \cite{cvc5}, Yices \cite{yices} und OpenSMT \cite{opensmt}.
Diese und viele andere nehmen jährlich an dem internationalen Wettbewerb SMT-COMP \cite{smtcomp} für SMT-Solver teil,
wo sie sich in verschiedenen Disziplinen miteinander messen.

Nach der Einführung in das Erfüllbarkeitsproblem SMT werden Anzahl-Constraints in SMT betrachtet,
bevor am Ende dieses Kapitels beleuchtet wird, wie SMT-Probleme nach Zielfunktionen optimiert werden können.

\section{Einführung in das Erfüllbarkeitsproblem SMT}
\label{sec:smtbasics}
Satisfiability Module Theories beschreibt das Erfüllbarkeitsproblem prädikatenlogischer Formeln unter spezifischen Theorien \cite{smt}.
Damit ermöglicht es im Gegensatz zum Erfüllbarkeitsproblem SAT, bei welchem ausschließlich boolesche Unbekannte aussagenlogisch verknüpft werden können,
die realitätsnahe Kodierung von Problemen wie beispielsweise Ressourcenzuordnungsproblemen \cite{rcpsp}.
Jene sind nachweislich stark NP-schwere kombinatorische Probleme, sodass sie nicht in polynomieller Zeit lösbar sind \cite{rcpspnp}.
Im Allgemeinen ist die Erfüllbarkeit prädikatenlogischer Formeln unentscheidbar \cite{smt}.
Die Eingrenzung von Syntax und Semantik (Interpretation von Funktionssymbolen) der Prädikatenlogik in Modellen von SMT-Problemen
ermöglicht die Entscheidbarkeit und erlaubt spezifische Lösungsverfahren unter bestimmten Hintergrund-Theorien (englisch \textit{background-theory}).
Diese Theorien werden zu SMT-Logiken wie beispielsweise \texttt{QF\_LRA} kombiniert.
Bei der Logik der quantorenfreien (QF) linearen (L) Real(R)-Arithmetik(A) sind das die Theorien der reellen Zahlen, linearen Arithmetik und Gleichheit/Ungleichheit.
Der diesjährige Sieger in jener Logik der SMT-COMP 2024 \texttt{OpenSMT} \cite{smtcomp2024results} implementiert als Lösungsverfahren eine Kombination aus CDCL(T) und modifiziertem Simplex \cite{opensmt}.
CDCL(T) is ein vollständiges Lösungsverfahren für prädikatenlogische Probleme unter einer Theorie \textit{T} und ersetzt alle Atome einer Formel mit booleschen Unbekannten \cite{smt}.
Dieses - nun SAT - Problem wird an einen SAT-Solver gegeben, welcher die Erfüllbarkeit prüft und für den Fall Erfüllbar eine erfüllende Belegung ermittelt.
Für jene Belegung überprüft ein (möglicherweise auch mehrere) Theorie-Solver dann deren Konsistenz in der Theorie \textit{T}.
Wenn diese Belegung \textit{T}-konsistent ist, so ist sie auch Modell für das prädikatenlogische Problem unter der Theorie \textit{T}.
Im Fall der \textit{T}-Inkonsistenz wird der SAT-Solver erneut aufgerufen.
Erwähnenswert sind hier zahlreiche integrierte Verfahren zur Effizienzsteigerung wie zum Beispiel das Lernen abgeleiteter Lemmas.

Für das Pressenplanungsproblem sind neben \texttt{QF\_LRA} auch \texttt{QF\_LIA} und \texttt{QF\_BV} relevante Logiken.
In der SMT-COMP 2024 siegte \texttt{OpenSMT} in den Logiken \texttt{QF\_LRA} und \texttt{QF\_LIA}, während \texttt{Bitwuzla} in der Logik \texttt{QF\_BV} den ersten Platz belegte \cite{smtcomp2024results}.

Die Kodierung der SMT-Probleme und damit Eingabe für die Solver ist im Standard SMTLib Version 2.6 festgehalten \cite{smtlib}.
Folgend die Darstellung relevanter Kernelemente des Sprachstandards anhand des Beispiels
$\exists x \in \mathbb{R}, \exists y \in \mathbb{R}: 5 = x + y$:

\begin{listing}[H]
    \inputminted[linenos=true]{bash}{Code/SMT/SMTLibSimple.smt2}
    \caption{Kodierung von $\exists x \in \mathbb{R}, \exists y \in \mathbb{R}: 5 = x + y$ in SMTLib-Syntax}
    \label{listing:smtlibsimple}
\end{listing}

In Zeile 1 wird der Solver angewiesen, im Fall der Erfüllbarkeit eine erfüllende Belegung zu erstellen.
Zeile zwei setzt die Logik für die darauffolgende Problemdeklaration.
In den Zeilen 4 und 5 werden Unbekannte (nullstellige Funktionen) der Sorte \texttt{Real} mit den Bezeichnern \texttt{x} und \texttt{y} angelegt.
Zeile 7 zeigt die Assertion eines Constraints auf den Unbekannten $x$ und $y$ und der Konstanten $5$.
Zeile 9 weist den Solver an, das deklarierte Problem auf Erfüllbarkeit zu überprüfen, während Zeile 10 für den Fall Erfüllbar ein Modell für das Problem ausgibt.
In Zeile 11 wird die Interaktion mit dem Solver beendet.

\section{SMT-Kodierung von Anzahl-Constraints}
\label{sec:smtcardinality}
Oftmals werden zur Kodierung von SMT-Problemen auch Anzahl-Constraints benötigt \cite{kovasznai}.
Auf einer Multimenge gegebener boolescher Variablen $B \in \dot{\mathbb{B}}$ soll dabei die Anzahl wahr belegter Variablen nach $k$ restriktiert werden.
Dabei existieren die folgenden drei Restriktionen:
\begin{itemize}
    \item Mindestens $k$: $\text{atLeast}: \mathbb{N}_0 \times \dot{\mathbb{B}} \rightarrow \mathbb{B}$
    \item Genau $k$: $\text{exactly}: \mathbb{N}_0 \times \dot{\mathbb{B}} \rightarrow \mathbb{B}$
    \item Höchstens $k$: $\text{atMost}: \mathbb{N}_0 \times \dot{\mathbb{B}} \rightarrow \mathbb{B}$
\end{itemize}

Anzahl-Constraints sind nicht Teil des SMTLib-Standards Version 2.6 \cite{smtlib}, sodass nur wenige Solver wie Z3 solche Kodierungen direkt unterstützen \cite{z3Cardinality}.
Daher wird zur manuellen Kodierung oft die ITE-Kodierung verwendet \cite{kovasznai}.
Dabei werden die einzelnen booleschen Unbekannten mithilfe eines If-Then-Else-Ausdrucks (kurz ITE) in numerische Ausdrücke umgewandelt, anschließend summiert und verglichen.
Mit der Hilfsfunktion $\text{count}: \dot{\mathbb{B}} \rightarrow \mathbb{N}_0$ lassen sich die Anzahl-Constraints einfach kodieren:

\[
    \begin{aligned}
        & \text{count}(B) = \sum_{\substack{b \in B}}
            \left(
                \begin{cases}
                    1 & \text{, } b \\
                    0 & \text{, sonst}
                \end{cases}
            \right) \\[5pt]
        & \text{atLeast}(k, B) = k \leq \text{count}(B) \\[5pt]
        & \text{exactly}(k, B) = k = \text{count}(B) \\[5pt]
        & \text{atMost}(k, B) = k \geq \text{count}(B)
    \end{aligned}
\]

Je nach gewählter Logik erfolgt die Summierung in der entsprechenden Sorte.
Ist für das SMT-Problem beispielsweise die Logik QF\_LRA gewählt, dann summiert $\text{count}(B)$ Terme der Sorte \texttt{Real}.
Dabei entstehen für jedes Anzahl-Constraint offensichtlich $n = \lvert B \rvert$ ITE-Ausdrücke, $n-1$ binäre Additionen und ein Vergleich.

Für bestimmte $k$ wie beispielsweise $k = 0$ oder $k = 1$ können Anzahl-Constraints allerdings von spezialisierten Kodierungen profitieren.
Einige davon sind folgende:
\begin{itemize}
    \item $\text{atLeast}(0,B) = \text{True}$
    \item $\text{atLeast}(1,B) = \bigvee\limits_{\substack{b \in B}} b $
    \item $\text{atMost}(0,B) = \neg\bigvee\limits_{\substack{b \in B}} b$
    \item $\text{atMost}(1,B) = \text{productEncoding}(B)$
    \item $\text{exactly}(0,B) = \neg\bigvee\limits_{\substack{b \in B}} b$
    \item $\text{exactly}(1,B) = \text{atLeast}(1,B) \land \text{atMost}(1,B)$
\end{itemize}

Die Produkt-Kodierung (englisch \textit{product-encoding}) ist eine aus SAT bekannte Kodierung für $\text{atMost}(1,B)$ \cite{amoChen}.
Dabei werden die $n = \lvert B \rvert$ Unbekannten in eine Matrix $M^{q \times p}$ mit $p = \lceil \sqrt {n} \rceil$ und $q = \lceil n/p \rceil$ gelegt.
Wenn $n < pq$ gilt, dann werden die übrigen Matrixeinträge mit \texttt{False} belegt.
Zusätzlich wird für jeden Zeilenindex $j \in \{0,\ldots,q-1\}$ eine Hilfsvariable $v_j$ und jeden Spaltenindex $i \in \{0,\ldots,p-1\}$ eine Hilfsvariable $u_i$ angelegt,
welche repräsentiert, ob in der jeweiligen Zeile $j$ oder Spalte $i$ mindestens ein $b \in B$ wahr ist.
Fasst man die Hilfsvariablen zu $U = \{ u_0,\ldots, u_{p-1} \}$ und $V = \{ v_0,\ldots, v_{q-1} \}$ zusammen, beschreibt folgende Gleichung,
dass höchstens eine der $n$ Unbekannten wahr ist:

\begin{multline*}
    \text{productEncoding}(B) = \\
    \text{atMost}(1,U) \land \text{atMost}(1,V) \land \bigwedge_{0 \leq i \leq p-1, 0 \leq j \leq q-1}^{0 \leq x \leq n-1, x = (i-1)(q-1)+j} ((\neg b_x \lor u_i) \land (\neg b_x \lor v_j))
\end{multline*}

Daraus folgen $2 \sqrt{n} + \mathcal{O}(\sqrt[4]{n})$ existentiell quantifizierte Hilfsvariablen und $2n + 4\sqrt{n} + \mathcal{O}(\sqrt[4]{n})$ Klauseln (konjugierte Disjunktionen).
Das ist die nachweisbar minimal notwendige Anzahl an Klauseln für das At-Most-One-Constraint \cite{amoChen, lowerBoundAMO}.

Aus softwaretechnischer Sicht und wie später in Abschnitt \ref{subsec:implCardinality} demonstriert, wäre eine Kodierung ohne Hilfsvariablen von Vorteil.
Die Produkt-Kodierung kann dazu einfach angepasst werden, indem alle Hilfsvariablen $u_i \in U$ und $v_j \in V$ folgend ersetzt werden:

\[
    u_i \mapsto \bigvee_{m_{j,i} \in M^{q \times p}, 0 \leq j \leq q-1} m_{j,i} \text{\quad und \quad}
    v_j \mapsto \bigvee_{m_{j,i} \in M^{q \times p}, 0 \leq i \leq p-1} m_{j,i}
\]

Das eliminiert die Hilfsvariablen und behält die Klauselanzahl bei.
Allerdings sind die entstehenden Klauseln um $\mathcal{O}(\sqrt{n})$ größer.

\section{Optimierung von SMT-Problemen}
Häufig sind bei SMT-Problemen zusätzlich auch spezifische Zielfunktionen von Interesse.
Diese haben besonders in der Industrie 4.0 eine große Relevanz \cite{omt}, so auch beim Pressenplanungsproblem.
Die Anzahl unverplanter Bauteile, eingesetzter Pressen und notwendigen Rests soll minimiert werden, siehe Abschnitt \ref{sec:optimierung}.

SMT-Probleme, bei denen zusätzlich nach bestimmten Zielfunktionen optimiert wird, fallen in eine neue Problemklasse,
jene der Optimization Module Theories (kurz OMT) \cite{omt}.
Genau wie die lineare Optimierung unterstützt auch OMT die Optimierung nach linearen Zielfunktionen.
Im Gegensatz zu ersterer, erlaubt OMT zusätzlich auch beliebige boolesche Kombinationen aus Atomen, wodurch die Modellierung komplexer Probleme vereinfacht wird.

Wenige SMT-Solver wie beispielsweise \texttt{Z3} unterstützen die direkte Kodierung von OMT-Zielfunktionen in SMTLib-Syntax \cite{nuz3}.
Folgend die Maximierung des Terms $x + y$ unter der Formel $\exists x \in \mathbb{Z}, \exists y \in \mathbb{Z}: x < 10 \land y < 5 \land (y < 7 \rightarrow x = 1)$:

\begin{listing}[H]
    \inputminted[linenos=true]{bash}{Code/SMT/OMTSimple.smt2}
    \caption{Maximierung von $x + y$ unter $\exists x \in \mathbb{Z}, \exists y \in \mathbb{Z}: x < 10 \land y < 5 \land (y < 7 \rightarrow x = 1)$}
    \label{listing:omtsimple}
\end{listing}

Mithilfe von Verfahren der linearen Optimierung ermittelt der Solver (hier \texttt{Z3}) korrekt die optimale Lösung $x \coloneqq 1, y \coloneqq 4$.