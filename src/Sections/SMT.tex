\chapter{Satisfiability Modulo Theories}
\label{chapter:smt}

In diesem Kapitel wird zuerst das Erfüllbarkeitsproblem der Satisfiability Module Theories - kurz SMT - vorgestellt.
Danach wird auf die Kernelemente des Sprachstandards SMTLib Version 2.6 \cite{smtlib} eingegangen.
Dieser ermöglicht die Kodierung von SMT-Problemen für SMT-Solver \cite{smt}.
Ein SMT-Solver - kurz Solver - ist ein Programm, welches für ein kodiertes Problem die Erfüllbarkeit dieses und für den
Fall \textit{erfüllbar} (englisch \textit{satisfiable} - kurz \textit{sat}) ein Modell für jenes bestimmen kann.
Einige bekannte Solver sind zum Beispiel Z3 \cite{z3}, CVC5 \cite{cvc5}, Yices \cite{yices} und OpenSMT \cite{opensmt}.
Diese und viele andere nehmen jährlich an dem internationalen Wettbewerb SMT-COMP \cite{smtcomp} für SMT-Solver teil,
wo sie sich in verschiedenen Disziplinen miteinander messen.
Nach dem Überblick über SMTLib werden Anzahl-Constraints in SMT betrachtet,
bevor am Ende dieses Kapitels beleuchtet wird, wie SMT-Probleme nach Zielfunktionen optimiert werden können.

\section{Einführung in das Erfüllbarkeitsproblem SMT}
Logiken, Sorten erklären.
Welche Logiken hier relevant?
Wie lösen die Solver die Probleme der Logiken?
Wie performant lösen die Solver diese - Komplexität?
Aktuelle SMT-Comp - beste Solver für relevante Kategorien nennen.

\section{Sprachstandard SMTLib Version 2.6}
Standard Version 2.6: \cite{smtlib}

\section{SMT-Kodierung von Anzahl-Constraints}
\label{sec:smtcardinality}
Oftmals werden zur Kodierung von SMT-Problemen auch Anzahl-Constraints benötigt \cite{kovasznai}.
Auf einer Menge von $n$ gegebenen booleschen Variablen $\{b_0, b_1, \ldots, b_{n-1}\}$ soll dabei die Anzahl wahr belegter Variablen nach $k$ restriktiert werden.
Dabei existieren die folgenden drei Restriktionen:
\begin{itemize}
    \item Mindestens $k$: $atLeast(k, \{b_0, b_1, \ldots, b_{n-1}\})$
    \item Genau $k$: $exactly(k, \{b_0, b_1, \ldots, b_{n-1}\})$
    \item Höchstens $k$: $atMost(k, \{b_0, b_1, \ldots, b_{n-1}\})$
\end{itemize}

Anzahl-Constraints sind nicht Teil des SMTLib-Standards Version 2.6 \cite{smtlib}, sodass nur wenige Solver wie Z3 solche Kodierungen direkt unterstützen \cite{z3Cardinality}.
Daher wird zur manuellen Kodierung oft die ITE-Kodierung verwendet \cite{kovasznai}.
Dabei werden die einzelnen booleschen Unbekannten mithilfe eines If-Then-Else-Ausdrucks (kurz ITE) in numerische Ausdrücke umgewandelt, anschließend summiert und verglichen.
Mithilfe der Hilfsfunktion $count(\{b_0, b_1, \ldots, b_{n-1}\})$ lassen sich die Anzahl-Constraints einfach kodieren:

\[
    \begin{aligned}
        & count(B) = \sum_{\substack{b \in B}}
            \left(
                \begin{cases}
                    1 & \text{, } b \\
                    0 & \text{, sonst}
                \end{cases}
            \right) \\[5pt]
        & atLeast(k, B) = k \leq count(B) \\[5pt]
        & exactly(k, B) = k = count(B) \\[5pt]
        & atMost(k, B) = k \geq count(B)
    \end{aligned}
\]

Je nach gewählter Logik erfolgt die Summierung in der entsprechenden Sorte.
Ist für das SMT-Problem beispielsweise die Logik QF\_LRA gewählt, dann summiert $count(B)$ Terme der Sorte \texttt{Real}.
Dabei entstehen für jedes Anzahl-Constraint offensichtlich $n$ ITE-Ausdrücke, $n-1$ binäre Additionen und ein Vergleich.

Für bestimmte $k$ wie beispielsweise $k = 0$ oder $k = 1$ können Anzahl-Constraints allerdings von spezialisierten Kodierungen profitieren.
Einige davon sind folgende:
\begin{itemize}
    \item $atLeast(0,B) = True$
    \item $atLeast(1,B) = \bigvee\limits_{\substack{b \in B}} b $
    \item $atMost(0,B) = \neg\bigwedge\limits_{\substack{b \in B}} b$
    \item $atMost(1,B) = productEncoding(1,B)$
    \item $exactly(0,B) = \neg\bigwedge\limits_{\substack{b \in B}} b$
    \item $exactly(1,B) = atLeast(1,B) \land atMost(1,B)$
\end{itemize}

Die Produkt-Kodierung (englisch \textit{product-encoding}) ist eine aus SAT bekannte Kodierung für $atMost(1,B)$ \cite{amoChen}.
Dabei werden die $n = \lvert B \rvert$ Unbekannten in eine Matrix $M^{q \times p}$ mit $p = \lceil \sqrt {n} \rceil$ und $q = \lceil n/p \rceil$ gelegt.
Das entspricht der Funktion $f: B \to M \quad$ mit $\quad f(b) = M_{j,i} \mapsto b$.
Zusätzlich wird für jeden Zeilenindex $j \in \{0,\ldots,q-1\}$ eine Hilfsvariable $v_j$ und jeden Spaltenindex $i \in \{0,\ldots,p-1\}$ eine Hilfsvariable $u_i$ angelegt,
welche repräsentiert, ob in der jeweiligen Zeile $j$ oder Spalte $i$ mindestens ein $b \in B$ wahr ist.
Fasst man die Hilfsvariablen zu $U = \{ u_0,\ldots, u_{p-1} \}$ und $V = \{ v_0,\ldots, v_{q-1} \}$ zusammen, beschreibt folgende Gleichung,
dass höchstens eine der $n$ Unbekannten wahr ist:

\begin{multline*}
    productEncoding(B) = \\
    atMost(1,U) \land atMost(1,V) \land \bigwedge_{0 \leq i \leq p-1, 0 \leq j \leq q-1}^{0 \leq x \leq n-1, x = (i-1)(q-1)+j} ((\neg b_x \lor u_i) \land (\neg b_x \lor v_j) )
\end{multline*}

Daraus folgen $2 \sqrt{n} + \mathcal{O}(\sqrt[4]{n})$ Hilfsvariablen und $2n + 4\sqrt{n} + \mathcal{O}(\sqrt[4]{n})$ Klauseln.
Das ist die nachweisbar minimale notwendige Anzahl an Klauseln für das At-Most-One-Constraint \cite{amoChen, lowerBoundAMO}.

% TODO: Forcing bei denen und bei mir?

Aus softwaretechnischer Sicht und wie später in Abschnitt \ref{subsec:implCardinality} demonstriert, wäre eine Kodierung ohne Hilfsvariablen von Vorteil.
Die Produkt-Kodierung kann dazu einfach angepasst werden, indem alle Hilfsvariablen $u_i \in U$ und $v_j \in V$ folgend ersetzt werden:

\[
    u_i \mapsto \bigvee_{m_{j,i} \in M^{q \times p}, 0 \leq j \leq q-1} m_{j,i} \text{\quad und \quad}
    v_j \mapsto \bigvee_{m_{j,i} \in M^{q \times p}, 0 \leq i \leq p-1} m_{j,i}
\]

Das eliminiert die Hilfsvariablen und behält die Klauselanzahl bei.
Allerdings sind die entstehenden Klauseln um $\mathcal{O}(\sqrt{n})$ größer.

\section{Optimierung von SMT-Problemen}
nuZ3 Paper hier sehr hilfreich \cite{nuz3}