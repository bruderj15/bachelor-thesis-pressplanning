\chapter{Zusammenfassung}
\label{chapter:zusammenfassung}
Pressenplanungsprobleme sind Ressourcenzuordnungsprobleme und diese sind NP-schwer.
Germanedge nutzt zur Lösung dieser bereits eine Heuristik, welche durch einige Vereinfachungen die Komplexität des Problems reduziert und somit
Lösungen nah am Optimum findet.
Diese Vereinfachungen ermöglichen eine schnelle ($< 10 \text{min}$), aber meist nicht optimale Lösung der Probleme.
Daher wurde in dieser Arbeit ein neuer Ansatz untersucht.

Das in dieser Arbeit entwickelte prototypische Programm implementiert einen vollständigen Lösungsansatz mit Methoden der Constraint-Programmierung.
Das Programm transformiert die Eingabe des Problems mithilfe der Haskell-Bibliothek \texttt{Hasmtlib} in ein in SMT-kodiertes Constraint-System,
welches anschließend durch einen externen SMT-Solver gelöst wird.
Dieser implementiert ein vollständiges Lösungsverfahren und garantiert somit die Optimalität der Lösung.

Ähnlich wie bei der Heuristik werden auch bei der Kodierung des Constraint-Systems Vereinfachungen vorgenommen.
Im Gegensatz zu ersterer behalten diese allerdings optimale Lösungen bei.
So werden beispielsweise Kodierungen mit verschiedenen aber festen Pressen-Anzahlen auf verschiedene CPU-Kerne aufgeteilt und parallel gelöst.
Unter diesen wird anschließend die beste Lösung gewählt.

Haskell und Bibliothek \texttt{Hasmtlib} eignen sich dabei aufgrund derer Ausdrucksstärke hervorragend für die Transformation der Eingabe zum Constraint-System.
Das Constraint-System wächst linear zur Größe der Eingabe.
Besonders Probleme kleiner und mittlerer Größe können schnell und optimal auf durchschnittlichen Maschinen gelöst werden.
Für das optimale Lösen größerer Probleme sind sowohl mehr Laufzeit als auch mehr Arbeitsspeicher notwendig.
Die Lösung mit der Constraint-Programmierung schlägt die Heuristik in knapp zwei Dritteln aller Testfälle dieser Arbeit.
Damit empfiehlt sich der neue Ansatz für die Anwendung in der Industrie bei kleinen und mittleren Pressenplanungsproblemen.
Bei den größten Probleminstanzen und der harten Grenze von zehn Minuten bleibt die Heuristik die bessere Wahl.

Das hier entwickelte Programm ist lediglich ein Prototyp, sodass einige Sonderfälle und spezielle Anforderungen
des Pressenplanungsproblems zukünftig noch abgedeckt werden sollten.
Sowohl Modellierung als auch Implementierung des Problems im entwickelten Programm lassen dabei noch Spielraum für Verbesserungen übrig.
Insgesamt zeigt diese Arbeit, dass die Constraint-Programmierung eine vielversprechende und vollständige
Alternative zur Lösung von Ressourcenzuordnungsproblemen durch Heuristiken bietet.
